\documentclass[12pt]{article}
\usepackage[utf8]{inputenc}
\usepackage[ukrainian]{babel}
\usepackage{amsmath, amsfonts}
\usepackage[T2A, T1]{fontenc}
\usepackage[margin = 0.5in, bottom = 1in]{geometry}
\usepackage{multirow, graphicx}
\usepackage{listings}

\lstset{
	basicstyle=\normalsize,
	breaklines = true,
	language = c++,
	breakatwhitespace = true,
	numbers = left,
	numberstyle = \tiny,
	columns = flexible,
	frame=tb,
	tabsize=4
}

\let\epsilon\varepsilon
\DeclareMathOperator\E{\bf E}

\makeatletter
\def\@bra#1#2\delims#3\fin{\kern-\nulldelimiterspace\left#1#3\right#2\kern-\nulldelimiterspace}
\newcommand\bra[2][()]{\@bra#1\delims#2\fin}
\newcommand\eq[3][..]{\begin{equation*}\bra[#1]{\begin{array}{#2}#3\end{array}}\end{equation*}}

\arraycolsep 2pt
\def\arraystretch{1.5}

\begin{document}
\centerline{Завдання 1}
Розв'язуємо стохастичне диференціальне рівняння
\eq{c}{dX_t = (\mu_1X_t + \mu_2)dt + (\sigma_1X_t + \sigma_2)dW_t.}
Покладемо $S_t = e^{at + bW_t}$ і застосуємо формулу Іто для добутку $X_tS_t$ (точніше, багатовимірну формулу Іто для $g = xe^{at+by}$ та $Y_t = (X_t, W_t)$) з урахуванням формули для $dX_t$
\eq{l}{
dX_t = (\mu_1X_t + \mu_2)dt + (\sigma_1X_t + \sigma_2)dW_t,\\
dX_tS_t = (a + \frac{b^2}{2})X_tS_tdt + S_tdX_t + bX_tS_tdW_t + bS_t(\sigma_1X_t + \sigma_2)dt =\\
= S_t\bra{\bra{(a + \frac{b^2 + 2b\sigma_1}{2} + \mu_1)X_t + \mu_2 + b\sigma_2}dt + \bra{(b + \sigma_1)X_t + \sigma_2}dW_t},}
тож, якщо покласти $b = -\sigma_1,\;a = -\mu_1 + \frac{\sigma_1^2}{2}$, то деякі доданки скоротяться і отримаємо
\eq{l}{
dX_te^{-(\mu_1 - \frac{\sigma_1^2}{2})t-\sigma_1W_t} = e^{-(\mu_1- \frac{\sigma_1^2}{2})t - \sigma_1W_t}\bra{(\mu_2 - \sigma_1\sigma_2)dt + \sigma_2dW_t},\\
X_t = e^{(\mu_1 - \frac{\sigma_1^2}{2})t + \sigma_1W_t}\bra{X_0 + \int_0^t{\!e^{-(\mu_1 - \frac{\sigma_1^2}{2})s-\sigma_1W_s}(\mu_2 - \sigma_1\sigma_2)\,ds} + \int_0^t{\!\sigma_2e^{-(\mu_1 - \frac{\sigma_1^2}{2})s - \sigma_1W_s}\,dW_s}}.
}
Врахуємо умови варіанту і отримаємо
\eq{c}{X_t = 1.9e^{3.5t + 0.8W_t},}
а наближення Ейлера будуватимуться за правилом
\eq{c}{X_{t+h} = 3.5X_th + 0.8X_t(W_{t+h}-W_t),\quad X_0 = 1.9.}
Просимулюємо наближений та точний розв'язок для різних діаметрів розбиття та часів зупинки $T$,
\resizebox{\textwidth}{!}{\includegraphics{./graphout.png}}\\
\resizebox{\textwidth}{!}{\includegraphics{./diffout.png}}\\

\begin{tabular}{*{5}{|c}|}
\hline
\multicolumn{2}{|c|}{}&\multicolumn{3}{c|}{Діаметр розбиття}\\\cline{3-5}
\multicolumn{2}{|c|}{}&$\frac{1}{10}$&$\frac{1}{100}$&$\frac{1}{1000}$\\\hline
\multirow{3}{*}{T}
&1
&\parbox[t]{0.25\textwidth}{Середнє: 24.5\\ Медіана: 13.6\\  $\sigma$:35.1\vspace{0.5\baselineskip}}
&\parbox[t]{0.25\textwidth}{Середнє: 4.6\\ Медіана: 2.05\\  $\sigma$:7.9\vspace{0.5\baselineskip}}
&\parbox[t]{0.25\textwidth}{Середнє: 0.95\\ Медіана: 0.55\\  $\sigma$:1.37\vspace{0.5\baselineskip}}\\\cline{2-5}

&10
&\parbox[t]{0.25\textwidth}{Середнє: $3.4\cdot10^{15}$\\ Медіана: $1.19\cdot10^{14}$\\  $\sigma$: $3.8\cdot10^{16}$\vspace{0.5\baselineskip}}
&\parbox[t]{0.25\textwidth}{Середнє: $1.4\cdot10^{15}$\\ Медіана: $4.15\cdot10^{13}$\\  $\sigma$: $1.72\cdot10^{16}$\vspace{0.5\baselineskip}}
&\parbox[t]{0.25\textwidth}{Середнє: $2.05\cdot10^{14}$\\ Медіана: $5.34\cdot10^{12}$\\  $\sigma$: $3.05\cdot10^{15}$\vspace{0.5\baselineskip}}\\\cline{2-5}

&50
&\parbox[t]{0.25\textwidth}{Середнє: $8.3\cdot10^{77}$\\ Медіана: $2.19\cdot10^{69}$\\  $\sigma$: $8.3\cdot10^{79}$\vspace{0.5\baselineskip}}
&\parbox[t]{0.25\textwidth}{Середнє: $9.3\cdot10^{74}$\\ Медіана: $1.9\cdot10^{69}$\\  $\sigma$: $3.77\cdot10^{76}$\vspace{0.5\baselineskip}}
&\parbox[t]{0.25\textwidth}{Середнє: $1.2\cdot10^{74}$\\ Медіана: $3.76\cdot10^{68}$\\  $\sigma$: $5.29\cdot10^{75}$\vspace{0.5\baselineskip}}\\\hline
\end{tabular}

З таблиці можна побачити, що відхилення зменшуються зі зменшенням діаметру розбиття. Це особливо помітно для $T=1$, а для інших його значень варто згадати, що, наприклад, $E X_{50} = e^{(3.5 - \frac{0.8^2}{2})50} = e^{159} > 10^{69}$, у порівнянні з чим відхилення не неймовірно великі.
Обчислення виконані за допомогою такого коду
\lstinputlisting[title=main]{./main.cpp}
\lstinputlisting[title=RandomProcess header]{./RandomProcess.h}
\lstinputlisting[title=RandomProcess]{./RandomProcess.cu}
\lstinputlisting[title=simulator]{./simulator.cu}
\centerline{Завдання 2}
Розглядаємо стохастичне диференціальне рівняння
\eq{c}{dX_t = \theta2\frac{X_t}{1 + X_t^2}dt + (1 _ 3\sin X_t)dW_t, \quad X_0 = 2,}
де $\theta=1$. Побудуємо для нього наближення Ейлера та оцінемо параметр за домопогою оцінки найбільшої вірогідності для різних діаметрів розбиття та довжин відрізка $T$. Отримуємо такий результат

\begin{tabular}{*{5}{|c}|}
\hline
\multicolumn{2}{|c|}{}&\multicolumn{3}{c|}{Діаметр розбиття}\\\cline{3-5}
\multicolumn{2}{|c|}{}&$\frac{1}{10}$&$\frac{1}{100}$&$\frac{1}{1000}$\\\hline
\multirow{3}{*}{T}
&10
&\parbox[t]{0.25\textwidth}{Середнє: 1.0072\\ Медіана: 1.0007\\  $\sigma$: 0.074\vspace{0.5\baselineskip}}
&\parbox[t]{0.25\textwidth}{Середнє: 1.0019\\ Медіана: 1.0002\\  $\sigma$: 0.042\vspace{0.5\baselineskip}}
&\parbox[t]{0.25\textwidth}{Середнє: 1\\ Медіана: 1\\  $\sigma$: 0.016\vspace{0.5\baselineskip}}\\\cline{2-5}

&20
&\parbox[t]{0.25\textwidth}{Середнє: 1.0023\\ Медіана: 1.0001\\  $\sigma$: 0.037\vspace{0.5\baselineskip}}
&\parbox[t]{0.25\textwidth}{Середнє: 1.0010\\ Медіана: 1\\  $\sigma$: 0.029\vspace{0.5\baselineskip}}
&\parbox[t]{0.25\textwidth}{Середнє: 1.0001\\ Медіана: 1\\  $\sigma$: 0.01\vspace{0.5\baselineskip}}\\\cline{2-5}

&50
&\parbox[t]{0.25\textwidth}{Середнє: 1.0006\\ Медіана: 1.0001\\  $\sigma$: 0.015\vspace{0.5\baselineskip}}
&\parbox[t]{0.25\textwidth}{Середнє: 1\\ Медіана: 1\\  $\sigma$: 0.018\vspace{0.5\baselineskip}}
&\parbox[t]{0.25\textwidth}{Середнє: 1\\ Медіана: 1\\  $\sigma$: 0.006\vspace{0.5\baselineskip}}\\\hline
\end{tabular}

Симуляцію здійснено за допомогою тієї ж програми, у яку внесено зміни, а саме замінено\\ $RandomProcess$ на $ParamProcess$
\lstinputlisting[title=ParamProcess header]{./ParamProcess.h}
\lstinputlisting[title=ParamProcess]{./ParamProcess.cu}

Рівняння має єдиний розв'язок, адже параметри (функції) ліпшицеві (бо диференційовні) та лінійного зросту (бо ще й неперервно диференційовні).
\end{document}