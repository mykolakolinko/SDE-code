\documentclass[12pt]{article}
\usepackage[utf8]{inputenc}
\usepackage[ukrainian]{babel}
\usepackage{amsmath, amsfonts}
\usepackage[T2A, T1]{fontenc}
\usepackage[margin = 0.5in, bottom = 1in]{geometry}
\usepackage{multirow, graphicx}
\usepackage{listings}

\lstset{
	basicstyle=\normalsize,
	breaklines = true,
	language = c++,
	breakatwhitespace = true,
	numbers = left,
	numberstyle = \tiny,
	columns = flexible,
	frame=tb,
	tabsize=4
}

\let\epsilon\varepsilon
\DeclareMathOperator\E{\bf E}

\makeatletter
\def\@bra#1#2\delims#3\fin{\kern-\nulldelimiterspace\left#1#3\right#2\kern-\nulldelimiterspace}
\newcommand\bra[2][()]{\@bra#1\delims#2\fin}
\newcommand\eq[3][..]{\begin{equation*}\bra[#1]{\begin{array}{#2}#3\end{array}}\end{equation*}}

\arraycolsep 2pt
\def\arraystretch{1.5}


\begin{document}
\begin{large}
\centerline{Колінько Микола. Варіант 7.}
\centerline{Домашня робота №3}
\end{large}
\vskip 1em
\centerline{Завдання 1}
Маємо стохастичне диференціальне рівняння
\eq{c}{dX_t = (\mu_1X_t + \mu_2)dt + (\sigma_1X_t + \sigma_2)dW_t.}
Застосуємо формулу Іто для добутку $X_tS_t$, де $S_t = e^{at + bW_t}$. Це багатовимірний випадок з фунцією $g = xe^{at+by}$ та $Y_t = (X_t, W_t)$. Рівняння для $dX_t$ вставимо в рівняння для $X_tS_t$.
\eq{l}{
dX_t = (\mu_1X_t + \mu_2)dt + (\sigma_1X_t + \sigma_2)dW_t,\\
dX_tS_t = (a + \frac{b^2}{2})X_tS_tdt + S_tdX_t + bX_tS_tdW_t + bS_t(\sigma_1X_t + \sigma_2)dt =\\
= S_t\bra{\bra{(a + \frac{b^2 + 2b\sigma_1}{2} + \mu_1)X_t + \mu_2 + b\sigma_2}dt + \bra{(b + \sigma_1)X_t + \sigma_2}dW_t},}
Покладемо $b = -\sigma_1,\;a = -\mu_1 + \frac{\sigma_1^2}{2}$, тоді деякі доданки скоротяться і отримаємо
\eq{l}{
dX_te^{-(\mu_1 - \frac{\sigma_1^2}{2})t-\sigma_1W_t} = e^{-(\mu_1- \frac{\sigma_1^2}{2})t - \sigma_1W_t}\bra{(\mu_2 - \sigma_1\sigma_2)dt + \sigma_2dW_t},\\
X_t = e^{(\mu_1 - \frac{\sigma_1^2}{2})t + \sigma_1W_t}\bra{X_0 + \int_0^t{\!e^{-(\mu_1 - \frac{\sigma_1^2}{2})s-\sigma_1W_s}(\mu_2 - \sigma_1\sigma_2)\,ds} + \int_0^t{\!\sigma_2e^{-(\mu_1 - \frac{\sigma_1^2}{2})s - \sigma_1W_s}\,dW_s}},\\
E[X_t] = e^{\mu_1 t}(x_0 - \frac{\mu_2}{\mu_1} (e^{-\mu_1 t}-1)).\\ \\
case \ \mu_1 = \sigma_2 = 0: X_t = x_0 e^{(\mu_1 - \frac{\sigma_1^2}{2})t + \sigma_1 W_t}, E[X_t] = x_0 e^{\mu_1 t}\\
case \ \mu_1 = \sigma_1 = 0: X_t = e^{\mu_1 t} \bra{x_0 + \sigma_2 \overset{t}{\underset{0}{\int}} e^{-\mu_1 s}dW_s},  E[X_t] = x_0 e^{\mu_1 t}\\
}
Підставимо коефіцієнти 19 варіанту: $x_0=2, \mu_1 = -1, \mu_2 = 0, \sigma_1 = -1.3, \sigma_2 = 0$
\eq{c}{X_t = 2e^{(-1 - \frac{(-1.3)^2}{2})t - 1.3 W_t},}
Вигляд наближення Ейлера для наших коефіцієнтів
\eq{c}{X_{t+dt} = -X_t dt - 1.3 X_t(W_{t+dt}-W_t),\quad X_0 = 2.}
Просимулюємо наближений та точний розв'язок для різних діаметрів розбиття $dt$ та часів зупинки $T$.
Нижче наведено графік однієї траекторії наближеного і точного розв'язку та графік усередненого і матсподівання точного розвязку.

\includegraphics[scale=0.42]{figure1.png}.
\includegraphics[scale=0.42]{figure2.png}.
\vskip 1 em
\begin{tabular}{*{5}{|c}|}
\hline
\multicolumn{2}{|c|}{}&\multicolumn{3}{c|}{Діаметр розбиття}\\\cline{3-5}
\multicolumn{2}{|c|}{}&$\frac{1}{10}$&$\frac{1}{100}$&$\frac{1}{1000}$\\\hline
\multirow{3}{*}{T}
&1
&\parbox[t]{0.25\textwidth}{Середнє: 1.9835\\ Медіана: 1.3105\\  $\sigma$: 2.2039\vspace{0.5\baselineskip}}
&\parbox[t]{0.25\textwidth}{Середнє: 2.1729\\ Медіана: 1.4052\\  $\sigma$: 2.4065\vspace{0.5\baselineskip}}
&\parbox[t]{0.25\textwidth}{Середнє: 2.2706\\ Медіана: 1.5615\\  $\sigma$: 2.2741\vspace{0.5\baselineskip}}\\\cline{2-5}

&10
&\parbox[t]{0.25\textwidth}{Середнє: 2.1688\\ Медіана: 1.4153\\  $\sigma$: 2.3235\vspace{0.5\baselineskip}}
&\parbox[t]{0.25\textwidth}{Середнє: 2.3485\\ Медіана: 1.4248\\  $\sigma$: 2.8303\vspace{0.5\baselineskip}}
&\parbox[t]{0.25\textwidth}{Середнє: 2.5618\\ Медіана: 1.5966\\  $\sigma$: 2.9760\vspace{0.5\baselineskip}}\\\cline{2-5}

&50
&\parbox[t]{0.25\textwidth}{Середнє: 2.1162\\ Медіана: 1.1300\\  $\sigma$: 2.5750\vspace{0.5\baselineskip}}
&\parbox[t]{0.25\textwidth}{Середнє: 2.9602\\ Медіана: 1.7926\\  $\sigma$: 4.1215\vspace{0.5\baselineskip}}
&\parbox[t]{0.25\textwidth}{Середнє: 2.0632\\ Медіана: 1.4073\\  $\sigma$:  2.0051\vspace{0.5\baselineskip}}\\\hline
\end{tabular}
\vskip 1em
Результати осаннього рядка для 100 траекторій, інакше програма працює дуже повільно. Всі інші результати для 1000 траекторій.
Обчислення виконані за допомогою програми мовою Matlab
\lstinputlisting{euler.m}
\lstinputlisting{init.m}
\end{document}
